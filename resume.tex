%!TEX TS-program = xelatex
%!TEX encoding = UTF-8 Unicode
% Awesome CV LaTeX Template
%
% This template has been downloaded from:
% https://github.com/posquit0/Awesome-CV
%
% Template Author:
% Claud D. Park <posquit0.bj@gmail.com>
% http://www.posquit0.com
%
% Template license:
% CC BY-SA 4.0 (https://creativecommons.org/licenses/by-sa/4.0/)
%
\usepackage{hyperref}


%%%%%%%%%%%%%%%%%%%%%%%%%%%%%%%%%%%%%%
%     Configuration
%%%%%%%%%%%%%%%%%%%%%%%%%%%%%%%%%%%%%%
%%% Themes: Awesome-CV
\documentclass[12pt, a4paper]{awesome-cv}

%%% Override a directory location for fonts(default: 'fonts/')
\fontdir[fonts/]

%%% Configure a directory location for sections
\newcommand*{\sectiondir}{resume/}

% Awesome Colors: awesome-emerald, awesome-skyblue, awesome-red, awesome-pink, awesome-orange
%                 awesome-nephritis, awesome-concrete, awesome-darknight
%% Color for highlight (no color hightlighting, keep it black)
\colorlet{awesome}{awesome-darknight}
% Define your custom color if you don't like awesome colors
%\definecolor{awesome}{HTML}{CA63A8}

%% Colors for text
%\definecolor{darktext}{HTML}{414141}
%\definecolor{text}{HTML}{414141}
%\definecolor{graytext}{HTML}{414141}
%\definecolor{lighttext}{HTML}{414141}

%%%%%%%%%%%%%%%%%%%%%%%%%%%%%%%%%%%%%%
%     3rd party packages
%%%%%%%%%%%%%%%%%%%%%%%%%%%%%%%%%%%%%%
%%% Needed to divide into several files
\usepackage{import}


%%%%%%%%%%%%%%%%%%%%%%%%%%%%%%%%%%%%%%
%     Personal Data
%%%%%%%%%%%%%%%%%%%%%%%%%%%%%%%%%%%%%%
%%% Essentials
\name{Liam Niehus-Staab} 
\position{}
\address{4940 Thunderbird Circle (Apt. 106), Boulder, CO 80303}

%%% Social
\mobile{(269) 491-4686}
\email{liamniehusstaab@gmail.com}
\homepage{niehusstaab.com}
\github{niehusst}
\linkedin{liam-niehus-staab} 



%%%%%%%%%%%%%%%%%%%%%%%%%%%%%%%%%%%%%%
%     Content
%%%%%%%%%%%%%%%%%%%%%%%%%%%%%%%%%%%%%%

\begin{document}
%%% Make a header for CV with personal data
\makecvheader

%%% contents

%---------------------------------------------------------------------------------------
%education
%---------------------------------------------------------------------------------------
\cvsection{Education}
\begin{cventries}
  \cventry
    {Bachelor of Arts}
    {Grinnell College}
    {Grinnell, IA}
    {Graduate of the Class of 2020}
    {
      \begin{cvitems}
      	\item {Computer Science Major}
      	\item {Major GPA: 3.76 \quad Overall GPA: 3.6}
        \item {Deans list Fall 2016 and Spring 2017}
      \end{cvitems}
    }
\end{cventries}

%---------------------------------------------------------------------------------------
%experience
%---------------------------------------------------------------------------------------
\cvsection{Experience}
\begin{cventries}

  %Honey intern
  \cventry
    {Honey (PayPal)}
    {Software Engineering Intern (Android)}
    {Boulder, CO}
    {June-September 2020}
    {
        \begin{cvitems}
            \item {Developed the team's first automated UI tests for major parts of the app.}
            \item {Refactored inaccessible views to be accessible for screen readers, keyboard navigation, and large font sizes.}
            \item {Built a screen to showcase our custom UI components, enabling greater reuse and easier tweaking of their functionality.}
        \end{cvitems}
    }

%hytrust intern
  \cventry
    {HyTrust Inc.}
    {Software Engineering Intern}
    {Mountain View, CA}
    {June 2019 - August 2019}
    {
      \begin{cvitems}
      	\item {Used openAPI to make public Django API self-documenting, greatly reducing documentation update turn-around time.}
	      \item {Increased products' network security by integrating SSL certificate validation between linked products in the field, utilizing openSSL.}
	      \item {Used Coverity to locate and fix potential bugs in API code.}
      \end{cvitems}
    }

%grammiegram android
  \cventry
    {GrammieGram LLC}
    {Head of Android Development}
    {Grinnell, IA}
    {August 2018 - June 2019}
    {
      \begin{cvitems}
        \item {Led a small team in an Agile environment with a test-driven approach.}
      	\item {Developed REST API in Python with Django for use in iOS and Android apps.}
	      \item {Created a proof-of-concept mobile app in Java that mirrors the messaging behavior of our web app.}
	      %\item {See the code at \underline{\href{https://github.com/niehusst/GrammieGram-Android}{https://github.com/niehusst/GrammieGram-Android}}}
      \end{cvitems}
    }


%MAP project
  \cventry
    {Undergraduate Research Project, Image Synthesis for Machine Learning}
    {Research Assistant}
    {Grinnell, IA}
    {June-August 2018}
    {
      \begin{cvitems}
        \item{Developed a C/C++ program to generate synthetic training data for a text recognizing neural network.}
        \item{Training with data generator improved model accuracy by 10\% over accuracy when trained on industry-standard data sets.}
	      \item{See the research code at \underline{\href{https://github.com/niehusst/MapTextSynthesizer}{https://github.com/niehusst/MapTextSynthesizer}}}
	    \item{See publication at \underline{\href{https://ieeexplore.ieee.org/abstract/document/8978121}{https://ieeexplore.ieee.org/abstract/document/8978121}}}
      \end{cvitems}
    }


%member of appdev grinnell
%  \cventry
%    {AppDev Grinnell}
%    {AppDev Junior Engineer}
%    {Grinnell, IA}
%    {January 2018 - Present}
%    {
%      \begin{cvitems}
%        \item {Perform quality assurance for the Grinnell Directory Android application.}%
%	\item {Wrote tests using JUnit, Mockito and Espresso to support a large code %base.}
%      \end{cvitems}
%    }


%grammiegram web app
  \cventry
    {GrammieGram LLC}
    {Co-Founder and Full Stack Developer}
    {Grinnell, IA}
    {April-August 2018}
    {
      \begin{cvitems}
        \item {Developed the backend for user preferences using Python and the Django framework.}
        \item {Constructed and maintained unit testing for the GrammieGram web app.}
        \item {Designed frontend UI for the contact management page using JavaScript and Bootstrap.}
        \item {Gained experience working remotely with a team using Git version control.}
	      \item {See the website at \underline{\href{https://grammiegram.com}{https://grammiegram.com}}}
      \end{cvitems}
    }



\end{cventries}


%---------------------------------------------------------------------------------------
%skills
%---------------------------------------------------------------------------------------
\cvsection{Skills}
\begin{cvskills}
  \cvskill
    {Programming Languages} 
    {Python, Java, C, C++} 

  \cvskill
    {Web}
    {Django, APIs}

  \cvskill
    {Mobile}
    {Android}

\end{cvskills}


%---------------------------------------------------------------------------------------
%projects section
%---------------------------------------------------------------------------------------
\cvsection{Projects}
\begin{cventries}

% HackUIowa Compliance Accomplice
    \cventry
	{HackUIowa}
 	{Compliance Accomplice}
	{Oct 2018}
	{}
	{
	  \begin{cvitems}
	     \item{Developed email middleware using Python to detect and prevent SEC compliance violations using Natural Language Processing.}
	     \item{Won “Best Commercial Potential” award from John Pappajohn Entrepreneurial Center and “Best use of Google Cloud” award from Major League Hacking.}
	     \item{See project at \underline{\href{https://github.com/niehusst/ComplianceAccomplice}{https://github.com/niehusst/ComplianceAccomplice}}} 
	  \end{cvitems}
	}

%MyRSA
%  \cventry
%    {}
%    {MyRSA}
%    {June 2019}
%    {}
%    {
%	\begin{cvitems}
%	    \item{Implemented RSA encryption from scratch in C, using openSSL calculations.}
%	    \item{Developed a TCP server using POSIX sockets for exchanging encrypted messages.}
%	\end{cvitems}
%    }

%frc programmer
%  \cventry
%    {First Robotics Competition, Team 6002}
%    {Robot Programmer}
%    {Kalamazoo, MI}
%    {Oct. 2015 - May 2016}
%    {
%      \begin{cvitems}
%        \item {Wrote Java code for the robot’s basic abilities and autonomous mode.}
%	       \item {Optimized PID controls to achieve smooth control of robot drive system.}
%      \end{cvitems}
%    }

\end{cventries}


%---------------------------------------------------------------------------------------
%relevant classes section
%---------------------------------------------------------------------------------------
%\cvsection{Relevant Course Work}
%\begin{cventries}
%
%coursera mobile robots
%computer vision
%
%\end{cventries}


\end{document}
