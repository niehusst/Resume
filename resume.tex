%!TEX TS-program = xelatex
%!TEX encoding = UTF-8 Unicode
% Awesome CV LaTeX Template
%
% This template has been downloaded from:
% https://github.com/posquit0/Awesome-CV
%
% Template Author:
% Claud D. Park <posquit0.bj@gmail.com>
% http://www.posquit0.com
%
% Template license:
% CC BY-SA 4.0 (https://creativecommons.org/licenses/by-sa/4.0/)
%
\usepackage{hyperref}


%%%%%%%%%%%%%%%%%%%%%%%%%%%%%%%%%%%%%%
%     Configuration
%%%%%%%%%%%%%%%%%%%%%%%%%%%%%%%%%%%%%%
%%% Themes: Awesome-CV
\documentclass[12pt, a4paper]{awesome-cv}

%%% Override a directory location for fonts(default: 'fonts/')
\fontdir[fonts/]

%%% Configure a directory location for sections
\newcommand*{\sectiondir}{resume/}

% Awesome Colors: awesome-emerald, awesome-skyblue, awesome-red, awesome-pink, awesome-orange
%                 awesome-nephritis, awesome-concrete, awesome-darknight
%% Color for highlight (no color hightlighting, keep it black)
\colorlet{awesome}{awesome-darknight}
% Define your custom color if you don't like awesome colors
%\definecolor{awesome}{HTML}{CA63A8}

%% Colors for text
%\definecolor{darktext}{HTML}{414141}
%\definecolor{text}{HTML}{414141}
%\definecolor{graytext}{HTML}{414141}
%\definecolor{lighttext}{HTML}{414141}

%%%%%%%%%%%%%%%%%%%%%%%%%%%%%%%%%%%%%%
%     3rd party packages
%%%%%%%%%%%%%%%%%%%%%%%%%%%%%%%%%%%%%%
%%% Needed to divide into several files
\usepackage{import}


%%%%%%%%%%%%%%%%%%%%%%%%%%%%%%%%%%%%%%
%     Personal Data
%%%%%%%%%%%%%%%%%%%%%%%%%%%%%%%%%%%%%%
%%% Essentials
\name{Liam Niehus-Staab} 
\position{}
%\address{4940 Thunderbird Circle (Apt. 106), Boulder, CO 80303}

%%% Social
\mobile{(269) 491-4686}
\email{liamniehusstaab@gmail.com}
\homepage{niehusstaab.com}
\github{niehusst}
\linkedin{liam-niehus-staab} 



%%%%%%%%%%%%%%%%%%%%%%%%%%%%%%%%%%%%%%
%     Content
%%%%%%%%%%%%%%%%%%%%%%%%%%%%%%%%%%%%%%

\begin{document}
%%% Make a header for CV with personal data
\makecvheader

%%% contents

%---------------------------------------------------------------------------------------
%experience
%---------------------------------------------------------------------------------------
\cvsection{Experience}
\begin{cventries}

  %Honey/PayPal
  \cventry
    {Honey (PayPal)}
    {Software Engineer II (iOS)}
    {Full Remote}
    {February 2021-Present}
    {
        \begin{cvitems}
            \item {Developed UI for product detail and checkout pages using Swift and Xcode for the Honey iOS app.}
            \item {Led the design and implementation of automated translation system for both mobile teams, saving ~6 hours of dev time each release.}
            \item {Implemented new toast notification UI, causing a 40\% increase in iOS checkout flow engagement.}
            \item {Helped port the Honey iOS app from React Native to native Swift.}
            \item {Created shared CI/CD distribution pipeline to make sharing SDKs between Honey and PayPal possible.}
            \item {Led multiple team culture/bonding activities, including regular bookclub and happy hour meetings.}
        \end{cvitems}
    }

  %Honey intern
  \cventry
    {Honey (PayPal)}
    {Software Engineering Intern (Android)}
    {Boulder, CO (remote)}
    {June 2020 - September 2020}
    {
        \begin{cvitems}
            \item {Fixed critical app accessibility issues for screen readers, keyboard navigation, and large font sizes}
            \item {Developed and maintained automated UI tests using Espresso and MockK for key parts of the app}
            \item {Built a showcase of custom UI components in Kotlin, enabling greater reuse and easier tweaking of their functionality}
        \end{cvitems}
    }

%hytrust intern
%  \cventry
%    {HyTrust Inc.}
%    {Software Engineering Intern}
%    {Mountain View, CA}
%    {June-August 2019}
%    {
%      \begin{cvitems}
%        \item {Used openAPI to make public Python Django API self-documenting, greatly accelerating documentation updates}
%              \item {Increased products' network security by integrating SSL certificate validation between linked cloud VMs, utilizing openSSL}
%              %\item {Used Coverity to locate and fix potential bugs in API code}
%      \end{cvitems}
%    }


%grammiegram android
  \cventry
    {GrammieGram LLC}
    {Head of Android Development}
    {Grinnell, IA}
    {August 2018 - June 2019}
    {
      \begin{cvitems}
	    \item {Created a proof-of-concept mobile app in Java that mirrored the behavior of the GrammieGram web app}
        \item {Led a team of three engineers in an Agile environment, with a test-driven approach}
      	\item {Developed REST API in Python with Django for use in iOS and Android apps}
	      %\item {See the code at \underline{\href{https://github.com/niehusst/GrammieGram-Android}{https://github.com/niehusst/GrammieGram-Android}}}
      \end{cvitems}
    }


%member of appdev grinnell
  \cventry
    {AppDev Club, Grinnell College}
    {Senior Member, Android}
    {Grinnell, IA}
    {January 2018 - May 2020}
    {
      \begin{cvitems}
        \item {Maintained the Grinnell-Directory Android app, used by staff and students of Grinnell College}	
        \item {Wrote tests using JUnit, Mockito and Espresso to support a large Java code base}
      \end{cvitems}
    }

%MAP project
%  \cventry
%    {Undergraduate Research Project, Image Synthesis for Machine Learning}
%    {Research Assistant}
%    {Grinnell, IA}
%    {June 2018 - August 2018}
%    {
%      \begin{cvitems}
%        \item{Developed a C/C++ program to generate synthetic training data for a text recognizing neural network}
%        \item{Training with our data generator improved model accuracy by 10\% over accuracy when trained on industry-standard data sets}
%	      \item{See the research code at \underline{\href{https://github.com/niehusst/MapTextSynthesizer}{https://github.com/niehusst/MapTextSynthesizer}}}
%	      \item{See publication at \underline{\href{https://ieeexplore.ieee.org/abstract/document/8978121}{https://ieeexplore.ieee.org/abstract/document/8978121}}}
%      \end{cvitems}
%    }


%grammiegram web app
%  \cventry
%    {GrammieGram LLC}
%    {Co-Founder and Full Stack Developer}
%    {Grinnell, IA}
%    {April-August 2018}
%    {
%      \begin{cvitems}
%        \item {Developed the backend for user preferences using Python and the Django framework}
%        \item {Constructed and maintained unit testing for the GrammieGram web app}
%        \item {Designed frontend UI for the contact management page using JavaScript and Bootstrap}
%        %\item {Gained experience working remotely with a team using Git version control}
%        %\item {See the website at \underline{\href{https://grammiegram.com}{https://grammiegram.com}}}
%      \end{cvitems}
%    }

\end{cventries}


%---------------------------------------------------------------------------------------
%projects section
%---------------------------------------------------------------------------------------
\cvsection{Projects}
\begin{cventries}

% partyq
  \cventry
    {(side project)}
    {partyq Android App}
    {Summer 2020 - Spring 2021}
    {}
    {
      \begin{cvitems}
        \item{Distributed music queue mobile app that connects guests via bluetooth and wifi and plays music through the Spotify dev SDK}
        \item{Developed in Kotlin using MVVM design pattern and released as open source}
        \item{Used Firebase for analytics + monitoring and beta release distribution}
        \item{Used GitHub Actions and Fastlane tools to automate releases to the Play Store}
        \item{See project at \underline{\href{https://github.com/niehusst/partyq}{https://github.com/niehusst/partyq}}}
        \item{See on Google Play Store at \underline{\href{https://play.google.com/store/apps/details?id=com.niehusst.partyq}{https://play.google.com/store/apps/details?id=com.niehusst.partyq}}}
      \end{cvitems}
    }

% HackUIowa Compliance Accomplice
%    \cventry
%	{HackUIowa}
% 	{Compliance Accomplice}
%	{October 2018}
%	{}
%	{
%	  \begin{cvitems}
%	     \item{Developed email middleware using Python to detect and prevent SEC compliance violations using Natural Language Processing}
%	     \item{Won “Best Commercial Potential” and “Best use of Google Cloud” awards}
%	     \item{See project at \underline{\href{https://github.com/niehusst/ComplianceAccomplice}{https://github.com/niehusst/ComplianceAccomplice}}} 
%	  \end{cvitems}
%	}


% FakeBlock
  % \cventry
  %   {(side project)}
  %   {FakeBlock}
  %   {Fall 2019}
  %   {}
  %   {
  %     \begin{cvitems}
  %       \item{Chrome browser extension for detecting and blocking fake news on Facebook through machine learning and fact-checking APIs}
  %       \item{See project at \underline{\href{https://fakeblocker.herokuapp.com/}{https://fakeblocker.herokuapp.com/}}}
  %     \end{cvitems}
  %   }

%MyRSA
%  \cventry
%    {}
%    {MyRSA}
%    {June 2019}
%    {}
%    {
%	\begin{cvitems}
%	    \item{Implemented RSA encryption from scratch in C, using openSSL calculations.}
%	    \item{Developed a TCP server using POSIX sockets for exchanging encrypted messages.}
%	\end{cvitems}
%    }

%frc programmer
%  \cventry
%    {First Robotics Competition, Team 6002}
%    {Robot Programmer}
%    {Kalamazoo, MI}
%    {Oct. 2015 - May 2016}
%    {
%      \begin{cvitems}
%        \item {Wrote Java code for the robot’s basic abilities and autonomous mode.}
%	       \item {Optimized PID controls to achieve smooth control of robot drive system.}
%      \end{cvitems}
%    }

\end{cventries}


%---------------------------------------------------------------------------------------
%education
%---------------------------------------------------------------------------------------
\cvsection{Education}
\begin{cventries}
  \cventry
    {Bachelor's Degree}
    {Grinnell College}
    {Grinnell, IA}
    {Graduate of the Class of 2020}
    {
      \begin{cvitems}
        \item {Computer Science Major}
        \item {Major GPA: 3.76 \quad Overall GPA: 3.6}
        %\item {Deans list Fall 2016 and Spring 2017}
      \end{cvitems}
    }
\end{cventries}


%---------------------------------------------------------------------------------------
%skills
%---------------------------------------------------------------------------------------
\cvsection{Skills}
\begin{cvskills}
  \cvskill
    {Programming Languages} 
    {Swift, Kotlin, Python, Java, JavaScript, Ruby} 

  \cvskill
    {Web}
    {Django, APIs}

  \cvskill
    {Mobile}
    {Android Native, iOS Native}

\end{cvskills}


%---------------------------------------------------------------------------------------
%relevant classes section
%---------------------------------------------------------------------------------------
%\cvsection{Relevant Course Work}
%\begin{cventries}
%
%coursera mobile robots
%computer vision
%
%\end{cventries}


\end{document}
